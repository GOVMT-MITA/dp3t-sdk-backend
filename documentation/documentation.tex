% Options for packages loaded elsewhere
\PassOptionsToPackage{unicode}{hyperref}
\PassOptionsToPackage{hyphens}{url}
%
\documentclass[a4paper
]{ubarticle}
\usepackage{graphicx}
\usepackage{amssymb,amsmath}
    \usepackage{parskip}
  \usepackage{tabu}
  \usepackage{longtable}
\makeatother

\usepackage{seqsplit}
\usepackage{hyperref}
\hypersetup{
  pdftitle={DPPPT API},
  hidelinks,
  pdfcreator={LaTeX via pandoc}}
\urlstyle{same} % disable monospaced font for URLs
\setlength{\emergencystretch}{3em} % prevent overfull lines
\providecommand{\tightlist}{%
  \setlength{\itemsep}{0pt}\setlength{\parskip}{0pt}}
\setcounter{secnumdepth}{-\maxdimen} % remove section numbering

\title{DPPPT API}
\date{\today}
\author{pepp-pt}

\begin{document}
\begin{titlepage}
	%\includegraphics[width=7cm]{ubique-logo.png}
	\hspace{4.3cm}
 {\raggedleft
	 \textbf{DP3T} \\
	%\hspace{11.5cm} Niederdorfstrasse 77 \\
%	8001 Zürich \\
\vspace{0.3cm}
	 
\par}
	\vspace{3cm}
	{\Huge COVIDAlert Malta API \par}
	\vspace{1.5cm}
	{\huge Documentation \par}
	\vspace{3cm}
	{ \large \today }
	\end{titlepage}
\thispagestyle{empty}
\clearpage
\tableofcontents
\clearpage
\include{introduction}
\part{Web Service}
\subsection{Introduction}
A test implementation is hosted on: https://data-test.ws.covidalert.gov.mt. 

This part of the documentation deals with the different API-Endpoints. Examples to fields are put into the models section~\ref{sec:Models} to increase readability. Every request lists all possible status codes and the reason for the status code.
\section{ /v1/gaen/ }
    \begin{verbatim}
    get /v1/gaen/
    \end{verbatim}
Hello return

\subsection{Responses}
\subsubsection{ 200 Success }
server live
 

    
        \begin{ubresponses}{\textwidth}{|Y|}
        \ubheader{Type}\\
        \hline
             \hyperref[sec:string] { string } \\
 \hline

        \end{ubresponses}
    
\section{ /v1/gaen/exposed }
    \begin{verbatim}
    post /v1/gaen/exposed
    \end{verbatim}
\begin{itemize}
Send exposed keys to server\item includes a fix for the fact that GAEN doesn&#39;t give access to the current day&#39;s exposed key
\end{itemize}
\subsection{ Request Headers }
\begin{ubparam}{\textwidth}{|H|c|Y|}
\ubheader{Field} & \ubheader{Type} & \ubheader{Description}\\
\hline
\ubheader{ User-Agent }   \textcolor{red}{\emph{*}}  &  string  & App Identifier (PackageName/BundleIdentifier) + App-Version + OS (Android/iOS) + OS-Version
 \\
\hline
\end{ubparam}

\subsection{ Request Body }
\begin{ubparam}{\textwidth}{|H|c|Y|}
\ubheader{Field} & \ubheader{Type} & \ubheader{Description}\\
\hline
\ubheader{  }   \textcolor{red}{\emph{*}}  &  GaenRequest  & The GaenRequest contains the SecretKey from the guessed infection date, the infection date itself, and some authentication data to verify the test result
 \\
\hline
\end{ubparam}
\subsection{Responses}
\subsubsection{ 200 Success }
The exposed keys have been stored in the database
 

    
        \begin{ubresponses}{\textwidth}{|Y|}
        \ubheader{Type}\\
        \hline
             \hyperref[sec:string] { string } \\
 \hline

        \end{ubresponses}
    
\subsubsection{ 400 Bad Request }
Invalid base64 encoding in GaenRequest
 


\subsubsection{ 403 Forbidden }
Authentication failed
 


\section{ /v1/gaen/exposednextday }
    \begin{verbatim}
    post /v1/gaen/exposednextday
    \end{verbatim}
Allows the client to send the last exposed key of the infection to the backend server. The JWT must come from a previous call to /exposed

\subsection{ Request Headers }
\begin{ubparam}{\textwidth}{|H|c|Y|}
\ubheader{Field} & \ubheader{Type} & \ubheader{Description}\\
\hline
\ubheader{ User-Agent }   \textcolor{red}{\emph{*}}  &  string  & App Identifier (PackageName/BundleIdentifier) + App-Version + OS (Android/iOS) + OS-Version
 \\
\hline
\end{ubparam}

\subsection{ Request Body }
\begin{ubparam}{\textwidth}{|H|c|Y|}
\ubheader{Field} & \ubheader{Type} & \ubheader{Description}\\
\hline
\ubheader{  }   \textcolor{red}{\emph{*}}  &  GaenSecondDay  & The last exposed key of the user
 \\
\hline
\end{ubparam}
\subsection{Responses}
\subsubsection{ 200 Success }
The exposed key has been stored in the backend
 

    
        \begin{ubresponses}{\textwidth}{|Y|}
        \ubheader{Type}\\
        \hline
             \hyperref[sec:string] { string } \\
 \hline

        \end{ubresponses}
    
\subsubsection{ 400 Bad Request }
\begin{itemize}
\item Ivnalid base64 encoded Temporary Exposure Key- TEK-date does not match delayedKeyDAte claim in Jwt
\end{itemize} 


\subsubsection{ 403 Forbidden }
No delayedKeyDate claim in authentication
 


\section{ /v1/gaen/exposed/{keyDate} }
    \begin{verbatim}
    get /v1/gaen/exposed/{keyDate}
    \end{verbatim}
Request the exposed key from a given date


\subsection{ Query Parameters }
\begin{ubparam}{\textwidth}{|H|c|Y|}
\ubheader{Field} & \ubheader{Type} & \ubheader{Description}\\
\hline
\ubheader{ publishedafter }   &  integer  & Restrict returned Exposed Keys to dates after this parameter. Given in milliseconds since Unix epoch (1970-01-01).
 \\
\hline
\end{ubparam}

\subsection{ Path Parameters }
\begin{ubparam}{\textwidth}{|H|c|Y|}
\ubheader{Field} & \ubheader{Type} & \ubheader{Description}\\
\hline
\ubheader{ keyDate }   \textcolor{red}{\emph{*}}  &  integer  & Requested date for Exposed Keys retrieval, in milliseconds since Unix epoch (1970-01-01). It must indicate the beginning of a TEKRollingPeriod, currently midnight UTC.
 \\
\hline
\end{ubparam}
\subsection{Responses}
\subsubsection{ 200 Success }
zipped export.bin and export.sig of all keys in that interval
 

        
    This request returns \textbf{ application/zip }. This represents zipped export.bin and export.sig of all keys in that interval
.

\subsubsection{ 500 Internal Server Error }
\begin{itemize}
\item invalid starting key date, doesn&#39;t point to midnight UTC- \emph{publishedAfter} is not at the beginning of a batch release time, currently 2h
\end{itemize} 


\section{ /v1/gaen/buckets/{dayDateStr} }
    \begin{verbatim}
    get /v1/gaen/buckets/{dayDateStr}
    \end{verbatim}
Request the available release batch times for a given day


\subsection{ Path Parameters }
\begin{ubparam}{\textwidth}{|H|c|Y|}
\ubheader{Field} & \ubheader{Type} & \ubheader{Description}\\
\hline
\ubheader{ dayDateStr }   \textcolor{red}{\emph{*}}  &  string  & Starting date for exposed key retrieval, as ISO-8601 format
 \\
\hline
\end{ubparam}
\subsection{Responses}
\subsubsection{ 200 Success }
zipped export.bin and export.sig of all keys in that interval
 

    
        \begin{ubresponses}{\textwidth}{|Y|}
        \ubheader{Type}\\
        \hline
             \hyperref[sec:DayBuckets] { DayBuckets } \\
 \hline

        \end{ubresponses}
    
\subsubsection{ 500 Internal Server Error }
invalid starting key date, points outside of the retention range
 


\section{ /v2/gaen/ }
    \begin{verbatim}
    get /v2/gaen/
    \end{verbatim}
Hello return

\subsection{Responses}
\subsubsection{ 200 Success }
server live
 

    
        \begin{ubresponses}{\textwidth}{|Y|}
        \ubheader{Type}\\
        \hline
             \hyperref[sec:string] { string } \\
 \hline

        \end{ubresponses}
    
\section{ /v2/gaen/exposed }
    \begin{verbatim}
    post /v2/gaen/exposed
    \end{verbatim}
    \begin{verbatim}
    get /v2/gaen/exposed
    \end{verbatim}
Requests keys published \emph{after} lastKeyBundleTag. The response includes also international keys if includeAllInternationalKeys is set to true. (default is false)

\subsection{ Request Headers }
\begin{ubparam}{\textwidth}{|H|c|Y|}
\ubheader{Field} & \ubheader{Type} & \ubheader{Description}\\
\hline
\ubheader{ User-Agent }   \textcolor{red}{\emph{*}}  &  string  & App Identifier (PackageName/BundleIdentifier) + App-Version + OS (Android/iOS) + OS-Version
 \\
\hline
\end{ubparam}

\subsection{ Query Parameters }
\begin{ubparam}{\textwidth}{|H|c|Y|}
\ubheader{Field} & \ubheader{Type} & \ubheader{Description}\\
\hline
\ubheader{ countries }   &  string[]  & List of countries of interest of requested keys. (iso-3166-1 alpha-2).
 \\
\hline
\ubheader{ lastKeyBundleTag }   &  integer  & Only retrieve keys published after the specified key-bundle tag. Optional, if no tag set, all keys for the retention period are returned
 \\
\hline
\end{ubparam}

\subsection{ Request Body }
\begin{ubparam}{\textwidth}{|H|c|Y|}
\ubheader{Field} & \ubheader{Type} & \ubheader{Description}\\
\hline
\ubheader{  }   \textcolor{red}{\emph{*}}  &  GaenV2UploadKeysRequest  & JSON Object containing all keys.
 \\
\hline
\end{ubparam}
\subsection{Responses}
\subsubsection{ 200 Success }
The exposed keys have been stored in the database
 

    
        \begin{ubresponses}{\textwidth}{|Y|}
        \ubheader{Type}\\
        \hline
             \hyperref[sec:string] { string } \\
 \hline

        \end{ubresponses}
    
\subsubsection{ 400 Bad Request }
\begin{itemize}
\item Invalid base64 encoding in GaenRequest- negative rolling period- fake claim with non-fake keys
\end{itemize} 


\subsubsection{ 403 Forbidden }
Authentication failed
 


\subsubsection{ 200 Success }
zipped export.bin and export.sig of all keys in that interval
 

        
    This request returns \textbf{ application/octet-stream }. This represents zipped export.bin and export.sig of all keys in that interval
.

\subsubsection{ 500 Internal Server Error }
Invalid \emph{lastKeyBundleTag}
 


\section{ /v2/gaen/exposed/raw }
    \begin{verbatim}
    get /v2/gaen/exposed/raw
    \end{verbatim}
Requests keys published \emph{after} lastKeyBundleTag. The response includes also international keys if includeAllInternationalKeys is set to true. (default is false)


\subsection{ Query Parameters }
\begin{ubparam}{\textwidth}{|H|c|Y|}
\ubheader{Field} & \ubheader{Type} & \ubheader{Description}\\
\hline
\ubheader{ countries }   &  string[]  & List of countries of interest of requested keys. (iso-3166-1 alpha-2).
 \\
\hline
\ubheader{ lastKeyBundleTag }   &  integer  & Only retrieve keys published after the specified key-bundle tag. Optional, if no tag set, all keys for the retention period are returned
 \\
\hline
\end{ubparam}
\subsection{Responses}
\subsubsection{ 200 Success }
List of all keys in that interval for the given countries of interest
 

    
        \begin{ubresponses}{\textwidth}{|Y|}
        \ubheader{Type}\\
        \hline
             \hyperref[sec:array[]] { array[] } \\
 \hline

        \end{ubresponses}
    
\subsubsection{ 500 Internal Server Error }
Invalid \emph{lastKeyBundleTag}
 



\part{Models}
All Models, which are used by the Endpoints are described here. For every field we give examples, to give an overview of what the backend expects.
\label{sec:Models}
\subsection{ DayBuckets }
\label{sec:DayBuckets}
\begin{ubresponses}{\textwidth}{|H|c|Y|p{3cm}|}
\ubheader{Field} & \ubheader{Type}  &\ubheader{Description}& \ubheader{Example}\\
\hline
 \ubheader{ dayTimestamp }  & \hyperref[sec:integer]{ integer }   & The day of all buckets, as midnight in milliseconds since the Unix epoch (1970-01-01)
 &  \seqsplit{1593043200000} \\
\hline
 \ubheader{ day }  & \hyperref[sec:string]{ string }   & The day as given by the request in /v1/gaen/buckets/{dayDateStr}
 &  \seqsplit{2020-06-27} \\
\hline
 \ubheader{ relativeUrls }  & \hyperref[sec:string]{ string[] }   & Relative URLs for the available release buckets
 &  \seqsplit{} \\
\hline

\end{ubresponses}

\subsection{ GaenKey }
\label{sec:GaenKey}
\begin{ubresponses}{\textwidth}{|H|c|Y|p{3cm}|}
\ubheader{Field} & \ubheader{Type}  &\ubheader{Description}& \ubheader{Example}\\
\hline
 \ubheader{ keyData }  \textcolor{red}{\emph{*}}  & \hyperref[sec:string]{ string }   & Represents the 16-byte Temporary Exposure Key in base64
 &  \seqsplit{} \\
\hline
 \ubheader{ rollingStartNumber }  \textcolor{red}{\emph{*}}  & \hyperref[sec:integer]{ integer }   & The ENIntervalNumber as number of 10-minute intervals since the Unix epoch (1970-01-01)
 &  \seqsplit{} \\
\hline
 \ubheader{ rollingPeriod }  \textcolor{red}{\emph{*}}  & \hyperref[sec:integer]{ integer }   & The TEKRollingPeriod indicates for how many 10-minute intervals the Temporary Exposure Key is valid
 &  \seqsplit{} \\
\hline
 \ubheader{ transmissionRiskLevel }  \textcolor{red}{\emph{*}}  & \hyperref[sec:integer]{ integer }   & According to the Google API description a value between 0 and 4096, with higher values indicating a higher risk
 &  \seqsplit{} \\
\hline
 \ubheader{ fake }  & \hyperref[sec:integer]{ integer }   & If fake = 0, the key is a valid key. If fake = 1, the key will be discarded.
 &  \seqsplit{} \\
\hline

\end{ubresponses}

\subsection{ GaenRequest }
\label{sec:GaenRequest}
\begin{ubresponses}{\textwidth}{|H|c|Y|p{3cm}|}
\ubheader{Field} & \ubheader{Type}  &\ubheader{Description}& \ubheader{Example}\\
\hline
 \ubheader{ gaenKeys }  \textcolor{red}{\emph{*}}  & \hyperref[sec:GaenKey]{ GaenKey[] }   & \begin{itemize}
Between 14 and 30 Temporary Exposure Keys\item zero or more of them might be fake keys. Starting with EN 1.5 it is possible that clients send more than 14 keys.
\end{itemize} &  \seqsplit{} \\
\hline
 \ubheader{ delayedKeyDate }  \textcolor{red}{\emph{*}}  & \hyperref[sec:integer]{ integer }   & Prior to version 1.5 Exposure Keys for the day of report weren't available (since they were still used throughout this day RPI=144), so the submission of the last key had to be delayed. This Unix timestamp in milliseconds specifies, which key date the last key (which will be submitted on the next day) will have. The backend then issues a JWT to allow the submission of this last key with specified key date. This should not be necessary after the Exposure Framework is able to send and handle keys with RollingPeriod < 144 (e.g. only valid until submission).
 &  \seqsplit{} \\
\hline
 \ubheader{ countries }  & \hyperref[sec:string]{ string[] }   & List of countries of interest.
 &  \seqsplit{} \\
\hline

\end{ubresponses}

\subsection{ GaenSecondDay }
\label{sec:GaenSecondDay}
\begin{ubresponses}{\textwidth}{|H|c|Y|p{3cm}|}
\ubheader{Field} & \ubheader{Type}  &\ubheader{Description}& \ubheader{Example}\\
\hline
 \ubheader{ delayedKey }  \textcolor{red}{\emph{*}}  & \hyperref[sec:GaenKey]{ GaenKey }   &  &  \seqsplit{} \\
\hline

\end{ubresponses}

\subsection{ GaenV2UploadKeysRequest }
\label{sec:GaenV2UploadKeysRequest}
\begin{ubresponses}{\textwidth}{|H|c|Y|p{3cm}|}
\ubheader{Field} & \ubheader{Type}  &\ubheader{Description}& \ubheader{Example}\\
\hline
 \ubheader{ countries }  & \hyperref[sec:string]{ string[] }   & List of countries of interest.
 &  \seqsplit{} \\
\hline
 \ubheader{ gaenKeys }  \textcolor{red}{\emph{*}}  & \hyperref[sec:GaenKey]{ GaenKey[] }   & \begin{itemize}
30 Temporary Exposure Keys\item zero or more of them might be fake keys.
\end{itemize} &  \seqsplit{} \\
\hline

\end{ubresponses}



\end{document}
